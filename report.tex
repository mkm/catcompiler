\documentclass{article}
\usepackage[utf8]{inputenc}
\usepackage{fancyhdr}
\usepackage[danish]{babel}
\usepackage{verbatim}
\usepackage{listings}
\pagestyle{fancyplain}
\author{Mikkel Kragh Mathiesen \& Rasmus Abrahamsen}
\title{Catcompiler}
\date{\today}
\lhead{Mikkel Kragh Mathiesen \& Rasmus Abrahamsen}
\rhead{\today}
%\write18{./iMake clean; ./iMake > out.tex}
\begin{document}
\maketitle

\section{Parsing}
% For Parser.grm skal der kort forklares hvordan grammatikken er gjort entydig (ved omskrivning eller brug af operatorpræcedenserklæringer) samt beskrivelse af eventuelle ikke-åbenlyse løsninger, f.eks. i forbindelse med opbygning af abstrakt syntaks. Det skal bemærkes, at alle konflikter skal fjernes v.h.a. præcedenserklæringer eller omskrivning af syntaks. Med andre ord må MosML-yacc ikke rapportere konflikter i tabellen.

Parsing of tuples with a single element has been problematic. We have solved it by matching comma separated expressions in parentheses. We then match said expressions with colon id.

We have a few shift/reduce conflicts which we have chosen to ignore as we have not been able to make a test case in which it makes a difference.

\section{Typechecking}
% For Type.sml skal kort beskrives, hvordan typerne checkes for de nye konstruktioner. Brug evt. en form, der ligner figur 6.2 i Basics of Compiler Design.

\begin{itemize}
	\item {\tt Cat.True} \& {\tt Cat.False} \\
	Returns {\tt bool}.
	\item {\tt Cat.Equal} \\
	The two expressions are evaluated to a type each. If both types are eiher {\tt int} or {\tt bool}, {\tt bool} will be returned.
	\item {\tt Cat.Less} \\
	Both expressions are evaluated and should be {\tt int} then {\tt bool} is returned.
	\item {\tt Cat.Not} \\
	The expression is evaluated and should be {\tt bool} and {\tt bool} is returned.
	\item {\tt Cat.And} \& {\tt Cat.Or} \\
	Both expressions should be {\tt bool} and the result will be {\tt bool}.
	\item {\tt Cat.If} \\
	The conditional expression should be {\tt bool} and the then-expression and else-expression should both have the same type.
	\item {\tt Cat.Let} \\
	The declarations are evaluated and a symbol table is returned, which is used when evaluating the expression.
	\item {\tt Cat.MkTuple} \\
	When creating a tuple, a list of expressions are received. These expressions are matched against the definition of the tuple type. Each element in the list is matched against each element in the definition of the tuple. If anything does not match or the lists have differing lengths, an exception is raised.
	\item {\tt Cat.Case} \\
	The built-in function {\tt checkMatch} is called and does everything we need.
	\item {\tt Cat.Null} \\
	Returns {\tt TyVar name} of the {\tt Null}.
\end{itemize}
Declarations support pattern-matching. Therefore each expression for each patter needs to be checked to see if they have the same type.

\section{Compiling}
% For Compiler.sml skal kort beskrives, hvordan kode genereres for de nye konstruktioner. Brug evt. en form, der ligner figur 7.3 i Basics of Compiler Design.
\begin{itemize}
	\item {\tt Cat.True } \\
	Returns $-1$. Every bit is set to $1$. A variable is true when every bit is $1$.
	\item {\tt Cat.False } \\
	Returns $0$. Every bit is $0$. 
	\item {\tt Cat.Null } \\
	Returns $0$.
	\item {\tt Cat.Equal } \\
	A new expression subtracting the expressions from each other is compiled and stored in a temporary variable which is then XOR-ed with $-1$.
	\item {\tt Cat.Less } \\
	The two expressions are compiled and stored into two variables. A Mips {\tt SLT}-instruction (set-less-than) is added to compare the two expressions.
	\item {\tt Cat.Not } \\
	An expression is compiled and the returning value is negated using the Mips {\tt XOR}-instruction which is the return value.
	\item {\tt Cat.And } \& {\tt Cat.Or } \\
	Both expressions are compiled and their return values are used as inputs in the Mips {\tt AND}- or {\tt OR}-instruction.
	\item {\tt Cat.Let } \\
	The {\tt let} expression is built using the {\tt case} expression. For example {\tt let x = z in y} is equivalent to {\tt case z of x => y}.
	\item {\tt Cat.If } \\
	The three expressions are compiled and their values bound to variables. The if-expression comes first, then comes a Mips {\tt BNE} instruction which branches if the boolean expression is not $0$. It then jumps to the then code.
	If it does not branch, it runs the else and then jumps to the ending label which is after any of the instructions built from the if expressions.
	\item {\tt Cat.MkTuple } \\
	The list of expressions is compiled and a name for each is kept. The names are then stored in a list along with their position in the list. The list is then stored in the heap. A pointer to the list is returned.
	\item {\tt Cat.Case } \\
	The expression is compiled and the name is sent to the {\tt compileMatch} function to do all the work.
\end{itemize}

\section{Results}

%\lstinputlisting{out.tex}

% Rapporten skal beskrive hvorvidt oversættelse og kørsel af eksempelprogrammer (jvf. afsnit 8) giver den forventede opførsel, samt beskrivelse af afvigelser derfra. Endvidere skal det vurderes, i hvilket omfang de udleverede testprogrammer er dækkende og der skal laves nye testprogrammer, der dækker de største mangler ved testen.

In order to narrow down problems, we have been making simplified test cases covering various expression types.

% Kendte mangler i typechecker og oversætter skal beskrives, og i det omfang det er muligt, skal der laves forslag til hvordan disse evt. kan udbedres.

The formal specification of the programming language does not specify on which types the equality expression is defined. Therefore we have implemented equality on both {\tt int} and {\tt bool}. We have not done so for tuples as none of the included test files.

\end{document}
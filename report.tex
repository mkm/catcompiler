\documentclass{article}
\usepackage[utf8]{inputenc}
\usepackage{fancyhdr}
\usepackage[danish]{babel}
\usepackage{verbatim}
\usepackage{listings}
\pagestyle{fancyplain}
\author{Mikkel Kragh Mathiesen \& Rasmus Abrahamsen}
\title{Catcompiler}
\date{\today}
\lhead{Mikkel Kragh Mathiesen \& Rasmus Abrahamsen}
\rhead{\today}
%\write18{./iMake clean; ./iMake > out.tex}
\write18{git diff initial Type.sml > /tmp/typediff.txt}
\begin{document}
\maketitle

\section{Parsing}
% For Parser.grm skal der kort forklares hvordan grammatikken er gjort entydig (ved omskrivning eller brug af operatorpræcedenserklæringer) samt beskrivelse af eventuelle ikke-åbenlyse løsninger, f.eks. i forbindelse med opbygning af abstrakt syntaks. Det skal bemærkes, at alle konflikter skal fjernes v.h.a. præcedenserklæringer eller omskrivning af syntaks. Med andre ord må MosML-yacc ikke rapportere konflikter i tabellen.

Parsing of tuples with a single element has been problematic. We have solved it by matching comma separated expressions in parentheses. We then match said expressions with colon id.

\begin{lstlisting}
Exps1 : Exp { [$1] }
  | Exp COMMA Exps1 { $1 :: $3 };

Exps2 : Exp COMMA Exps1 {$1 :: $3 };

ParenExp : LPAR Exp RPAR { $2 };

Tuple : ParenExp COLON ID { Cat.MkTuple ([$1], #1 $3, $2) }
  | LPAR Exps2 RPAR COLON ID { Cat.MkTuple ($2, #1 $5, $1) };
\end{lstlisting}

We have a few shift/reduce conflicts which we have chosen to ignore as we have not been able to make a test case in which it makes a difference.

\section{Typechecking}
% For Type.sml skal kort beskrives, hvordan typerne checkes for de nye konstruktioner. Brug evt. en form, der ligner figur 6.2 i Basics of Compiler Design.

\begin{itemize}
	\item {\tt Cat.True} \& {\tt Cat.False} \\
	Returns {\tt bool}.
	\item {\tt Cat.Equal} \\
	The two expressions are evaluated to a type each. If both types are eiher {\tt int} or {\tt bool}, {\tt bool} will be returned.
	\begin{lstlisting}
| Cat.Equal (e1, e2, pos) =>
  (case (checkExp e1 vtable ftable ttable,
            checkExp e2 vtable ftable ttable) of
          (Int,Int) => Bool
       |(Bool,Bool) => Bool
       |	(TyVar s1, TyVar s2) => 
		    if s1 = s2 
		    then Bool 
		    else raise Error("Comparison of diff. types ",pos)
       | _ => raise Error("Comparison of differing types ",pos))
	\end{lstlisting}
	\item {\tt Cat.Less} \\
	Both expressions are evaluated and should be {\tt int} then {\tt bool} is returned.
	\begin{lstlisting}
| Cat.Less (e1, e2, pos) =>
  (case (checkExp e1 vtable ftable ttable,
         checkExp e2 vtable ftable ttable) of
          (Int,Int) => Bool
        | _ => raise Error ("Non-int argument to <",pos))
	\end{lstlisting}
	\item {\tt Cat.Not} \\
	The expression is evaluated and should be {\tt bool} and {\tt bool} is returned.
	\begin{lstlisting}
| Cat.Not (e1, pos) =>
    (case (checkExp e1 vtable ftable ttable) of
	      Bool => Bool
        | _ => raise Error ("Not has to be called with a boolean",pos))
	\end{lstlisting}
	\item {\tt Cat.And} \& {\tt Cat.Or} \\
	Both expressions should be {\tt bool} and the result will be {\tt bool}.
	\begin{lstlisting}
| Cat.And (e1,e2,pos) =>
  (case (checkExp e1 vtable ftable ttable,
         checkExp e2 vtable ftable ttable) of
       (Bool,Bool) => Bool
     | _ => raise Error ("And needs two booleans",pos))
| Cat.Or (e1,e2,pos) =>
  (case (checkExp e1 vtable ftable ttable,
         checkExp e2 vtable ftable ttable) of
       (Bool,Bool) => Bool
     | _ => raise Error ("Or needs two booleans",pos))
	\end{lstlisting}
	\item {\tt Cat.If} \\
	The conditional expression should be {\tt bool} and the then-expression and else-expression should both have the same type.
	\begin{lstlisting}
| Cat.If (ec, e1, e2, pos) =>
  let val (ct, t1, t2) = (
         checkExp ec vtable ftable ttable,
         checkExp e1 vtable ftable ttable, 
         checkExp e2 vtable ftable ttable)
   in
     if t1 = t2 andalso ct = Bool
     then t1
     else raise Error ("Or needs two booleans",pos)
   end
	\end{lstlisting}
	\item {\tt Cat.Let} \\
	The declarations are evaluated and a symbol table is returned, which is used when evaluating the expression.
	\begin{lstlisting}
| Cat.Let(d, e, _) =>
let
    val vtable' = checkDec d vtable ftable ttable
in
    checkExp e vtable' ftable ttable
end
	\end{lstlisting}
	\item {\tt Cat.MkTuple} \\
	When creating a tuple, a list of expressions are received. These expressions are matched against the definition of the tuple type. Each element in the list is matched against each element in the definition of the tuple. If anything does not match or the lists have differing lengths, an exception is raised.
	\begin{lstlisting}
| Cat.MkTuple (exps, typname, pos) =>
 let
  fun x(exp::exps, (typ::typs) : CType list, pos, 
	vtable, ftable, ttable) = 
   let
    val exptyp : CType = checkExp exp vtable 
		ftable ttable
   in
    if exptyp = typ
    then x(exps, typs, pos, vtable, ftable, 
		ttable)
    else raise Error ("Tuple of wrong type, 
		expected "^ typeName exptyp ^" got " 
		^ typeName typ,pos)
   end
  | x([], [], pos, _, _, _) = ()
  | x(_,   _, pos, _, _, _) = raise Error 
	("Tuple of wrong type",pos)
 in
    (case lookup typname ttable of
         SOME (typs : CType list) => (x(exps, 
typs, pos, vtable, ftable, ttable); TyVar (typname))
        |       _ => raise Error ("Unknown type "^typname,pos))
 end
	\end{lstlisting}
	\item {\tt Cat.Case} \\
	The built-in function {\tt checkMatch} is called and does everything we need.
	\begin{lstlisting}
| Cat.Case(e, m, pos) => checkMatch m 
(checkExp e vtable ftable ttable) vtable 
ftable ttable pos
	\end{lstlisting}
	\item {\tt Cat.Null} \\
	Returns {\tt TyVar name} of the {\tt Null}.
\end{itemize}
Declarations support pattern-matching. Therefore each expression for each patter needs to be checked to see if they have the same type.

\begin{lstlisting}
and checkDec [] vtable ftable ttable = vtable
  | checkDec ((pat, exp, pos)::xs) vtable ftable ttable =
    let
        val expT = checkExp exp vtable ftable ttable
        val vtable' = checkPat pat expT ttable pos
    in
        checkDec xs (combineTables vtable' vtable pos) ftable ttable
    end
\end{lstlisting}

\section{Compiling}
% For Compiler.sml skal kort beskrives, hvordan kode genereres for de nye konstruktioner. Brug evt. en form, der ligner figur 7.3 i Basics of Compiler Design.
\begin{itemize}
	\item {\tt Cat.True } \\
	Returns $-1$. Every bit is set to $1$. A variable is true when every bit is $1$.
	\item {\tt Cat.False } \\
	Returns $0$. Every bit is $0$. 
	\item {\tt Cat.Null } \\
	Returns $0$.
	\item {\tt Cat.Equal } \\
	A new expression subtracting the expressions from each other is compiled and stored in a temporary variable which is then XOR-ed with $-1$.
	\begin{lstlisting}
| Cat.Equal (e1,e2,pos) =>
  let
    val lbl1 = "_equaljump_" ^ newName ()
    val lbl2 = "_equaljump_" ^ newName ()
    val name1 = "_equalarg_" ^ newName ()
    val name2 = "_equalarg_" ^ newName ()
  in
    compileExp e1 vtable name1 @
    compileExp e2 vtable name2 @
    [
     Mips.BEQ (name1, name2, lbl1),
     Mips.LI (place, makeConst 0),
     Mips.J lbl2,
     Mips.LABEL lbl1,
     Mips.LI (place, makeConst (~1)),
     Mips.LABEL lbl2
    ]
  end
	\end{lstlisting}
	\item {\tt Cat.Less } \\
	The two expressions are compiled and stored into two variables. A Mips {\tt SLT}-instruction (set-less-than) is added to compare the two expressions.
	\begin{lstlisting}
| Cat.Less (e1,e2,pos) =>
  let
    val t1 = "_less1_"^newName()
    val t2 = "_less2_"^newName()
    val code1 = compileExp e1 vtable t1
    val code2 = compileExp e2 vtable t2
  in
    code1 @ code2 @ [Mips.SLT (place,t1,t2)]
  end
	\end{lstlisting}
	\item {\tt Cat.Not } \\
	An expression is compiled and the returning value is negated using the Mips {\tt XOR}-instruction which is the return value.
	\begin{lstlisting}
| Cat.Not (e, pos) =>
  let
      val t1 = "_not1_"^newName()
      val code1 = compileExp e vtable t1
      val t2 = "_not2_"^newName()
      val code2 = compileExp (Cat.Num(~1,pos)) vtable t2
  in
      code1 @ code2 @ [Mips.XOR (place, t1, t2)]
  end
	\end{lstlisting}
	\item {\tt Cat.And } \& {\tt Cat.Or } \\
	Both expressions are compiled and their return values are used as inputs in the Mips {\tt AND}- or {\tt OR}-instruction.
	\begin{lstlisting}
    | Cat.And (e1, e2, pos) => 
		compileExp (Cat.If(
			e1, 
			Cat.If(
				e2, 
				Cat.True(pos), 
				Cat.False(pos), pos), 
				Cat.False(pos), pos)) vtable place
    | Cat.Or (e1, e2, pos) =>
      compileExp (Cat.If(
		e1, 
		Cat.True(pos), 
		Cat.If(
			e2, 
			Cat.True(pos), 
			Cat.False(pos), pos), pos)) vtable place
	\end{lstlisting}
	\item {\tt Cat.Let } \\
	The {\tt let} expression is built using the {\tt case} expression. For example {\tt let x = z in y} is equivalent to {\tt case z of x => y}.
	\begin{lstlisting}
| Cat.Let ([], expResult, _) => compileExp expResult 
	vtable place
| Cat.Let ((patBind, expBind, _)::ds, expResult, pos) =>
  let
    val rest = Cat.Let (ds, expResult, pos)
    val r = Cat.Case (expBind, [(patBind, rest)], pos)
  in
    compileExp r vtable place
  end
	\end{lstlisting}
	\item {\tt Cat.If } \\
	The three expressions are compiled and their values bound to variables. The if-expression comes first, then comes a Mips {\tt BNE} instruction which branches if the boolean expression is not $0$. It then jumps to the then code.
	If it does not branch, it runs the else and then jumps to the ending label which is after any of the instructions built from the if expressions.
	\begin{lstlisting}
| Cat.If (e1,e2,e3,pos) =>
  let
    val if_ = "_if_"^newName()
    val then_ = "_then_"^newName()
    val end_ = "_end_"^newName()
    val if_code   = compileExp e1 vtable if_
    val then_code = compileExp e2 vtable place
    val else_code = compileExp e3 vtable place
  in
    if_code @ [Mips.BNE (if_, "0", then_)] 
    @ else_code
    @ [Mips.J(end_)]
    @ [Mips.LABEL(then_)] @ then_code
    @ [Mips.LABEL(end_)]
  end
	\end{lstlisting}
	\item {\tt Cat.MkTuple } \\
	The list of expressions is compiled and a name for each is kept. The names are then stored in a list along with their position in the list. The list is then stored in the heap. A pointer to the list is returned.
	\begin{lstlisting}
| Cat.MkTuple (es, t, pos) =>
  let
    val fieldCount = length es
    val places = List.tabulate (fieldCount, 
	fn _ => "_tuple_" ^ newName ())
    val fields = map (fn (e, place) => compileExp 
e vtable place) (ListPair.zip (es, places))
    val loads = map (fn (i, place) => Mips.SW 
(place, HP, makeConst (i * 4))) (ListPair.zip 
	(range 0 (fieldCount - 1), places))
  in
    List.concat fields @
    [Mips.ADDI (HP, HP, makeConst (fieldCount * 4))] @
    loads @
    [Mips.MOVE (place, HP)]
  end
	\end{lstlisting}
	\item {\tt Cat.Case } \\
	The expression is compiled and the name is sent to the {\tt compileMatch} function to do all the work.
	\begin{lstlisting}
| Cat.Case (exp, matches, pos) =>
  let
    val argPlace = "_casearg_" ^ newName ()
    val argCode = compileExp exp vtable argPlace
    val endLabel = "_caseend_" ^ newName ()
    val failLabel = "_Error_"
  in
    argCode @
    compileMatch matches argPlace 
place endLabel failLabel vtable @
    [Mips.LABEL endLabel]
  end
	\end{lstlisting}
\end{itemize}

\section{Results}

%\lstinputlisting{out.tex}

% Rapporten skal beskrive hvorvidt oversættelse og kørsel af eksempelprogrammer (jvf. afsnit 8) giver den forventede opførsel, samt beskrivelse af afvigelser derfra. Endvidere skal det vurderes, i hvilket omfang de udleverede testprogrammer er dækkende og der skal laves nye testprogrammer, der dækker de største mangler ved testen.

In order to narrow down problems, we have been making simplified test cases covering various expression types.

The compiler works for all included test cases including our own.

% Kendte mangler i typechecker og oversætter skal beskrives, og i det omfang det er muligt, skal der laves forslag til hvordan disse evt. kan udbedres.

The formal specification of the programming language does not specify on which types the equality expression is defined. Therefore we have implemented equality on both {\tt int} and {\tt bool}. We have not done so for tuples as none of the included test files.

\end{document}
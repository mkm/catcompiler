\documentclass{article}
\usepackage[utf8]{inputenc}
\usepackage{fancyhdr}
\usepackage[danish]{babel}
\pagestyle{fancyplain}
\author{Mikkel Kragh Mathiesen \& Rasmus Abrahamsen}
\title{Catcompiler}
\date{\today}
\lhead{Mikkel Kragh Mathiesen \& Rasmus Abrahamsen}
\rhead{\today}
\begin{document}
\maketitle

\section{Parsing}
% For Parser.grm skal der kort forklares hvordan grammatikken er gjort entydig (ved omskrivning eller brug af operatorpræcedenserklæringer) samt beskrivelse af eventuelle ikke-åbenlyse løsninger, f.eks. i forbindelse med opbygning af abstrakt syntaks. Det skal bemærkes, at alle konflikter skal fjernes v.h.a. præcedenserklæ- ringer eller omskrivning af syntaks. Med andre ord må MosML-yacc ikke rappor- tere konflikter i tabellen.

\section{Typechecking}
% For Type.sml skal kort beskrives, hvordan typerne checkes for de nye konstruktioner. Brug evt. en form, der ligner figur 6.2 i Basics of Compiler Design.

\begin{itemize}
	\item Cat.True
	\item Cat.False
	\item Cat.Equal
	\item Cat.Less
	\item Cat.Not
	\item Cat.And
	\item Cat.Or
	\item Cat.If
	\item Cat.Let
	\item Cat.MkTuple
	\item Cat.Case
	\item Cat.Null
\end{itemize}

\section{Compiling}
% For Compiler.sml skal kort beskrives, hvordan kode genereres for de nye konstruktioner. Brug evt. en form, der ligner figur 7.3 i Basics of Compiler Design.

\section{Results}
% Rapporten skal beskrive hvorvidt oversættelse og kørsel af eksempelprogram- mer (jvf. afsnit 8) giver den forventede opførsel, samt beskrivelse af afvigelser der- fra. Endvidere skal det vurderes, i hvilket omfang de udleverede testprogrammer er dækkende og der skal laves nye testprogrammer, der dækker de største mangler ved testen.

% Kendte mangler i typechecker og oversætter skal beskrives, og i det omfang det er muligt, skal der laves forslag til hvordan disse evt. kan udbedres.

\end{document}